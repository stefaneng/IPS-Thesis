In an absorbing Markov chain with exactly one absorbing state the time until absorption is called a phase-type distribution.
We touch briefly on the background of phase-type distributions, first studied by \cite{neuts1975}, including some interesting theorems on closure properties.
For readers interested in phase-type distributions a comprehensive account is found in \cite{neuts1981}.

Interacting particle systems are stochastic processes with a state space being graphs with different values on the nodes.
These have been studied since the 1970's with seminal work by Liggett published in 1985, and much work has been done since then.
We focus on the case of two specific particle systems, the voter model and the contact process, and restrict ourselves mainly to the case where the number of nodes in the graph is finite.
A short background on these interacting particle systems are included with a few key theorems including voter model clustering.
Simulations are shown to illustrate the behavior of the different models.

We focus on the finite case where we can find the exact distribution of the absorption time for these models.
Other work has been done on finite models such as \cite{cox1989} and \cite{durrett1988}.
In these works the behavior of the processes are often computed for fixed parameters with the number of nodes going to infinity.
In contrast, we fix the number of nodes and vary the infection rate in the contact process.

We start with the voter model on a complete graph and first find the phase-type distribution for the $n = 2,3,4$ cases.
The expected value is computed in the general case with $n$ nodes.
For the contact process we find the exact form of the phase-type distribution for the $n = 2$ case as a function of the infection rate.
The main result for the contact process is showing that the limiting distribution of the $n$ node complete graph, when scaled appropriately, tends to an exponential distribution with rate 1.
We also numerically compute the phase-type distributions of the complete graph with 3 nodes, a line with $n = 2, 3$, and torus with $n = 2,3,4$ nodes.
The expected values are graphed and compared.

A student with an introductory course in probability (understanding of random variables, probability spaces, distributions, etc.) should be able to follow this thesis as we include all relevant definitions and background.



