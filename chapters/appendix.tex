\section{Simulation}

\subsection{Simulation of contact process on \texorpdfstring{$\Z^2$}{Z2}}
In this section we describe how to simulate the linear voter model on the two-dimensional lattice $\Z^2$.
By nature of the simulation we have to restrict $\Z^2$ to some finite subset of $\Z^2$.
A new issue then arises with the boundaries.
If we want to simulate the behavior on $\Z^2$ then it is better to use \textit{periodic boundary conditions} where we let the state space be $\{0,1\}^{(\Z/n)^2}$ where $n$ is some finite number.
This means that each node on the graph will have exactly 4 neighbors.
We can also simulate a finite graph where the boundary simply has fewer edges than the other nodes.

The dynamics of the system are that each 1 waits exponential time with rate 1 and changes to 0.
Each 1 waits an independent exponential time with rate $\lambda$ and then places a 1 onto one of the $d$ neighbors with probability $1/2^d$.
If there already is a 1 at this spot then nothing happens.

\subsection{Simulation of Linear Voter Model on \texorpdfstring{$\Z^2$}{Z2}}

The behavior of the linear voter model can be described as each voter waits exponential time and then takes the opinion of one of its neighbors according to the transition probabilities.
Assume that we are simulating on an $N \times N$ grid.
Using the exponential scaling property, this behavior is equivalent to waiting exponential time with rate $N^2$ and selecting a voter at random and updating their opinion.

\subsection{Source Code}

All of the R source code is available at \href{https://github.com/stefaneng/IPS-Thesis}{github.com/stefaneng/IPS-Thesis}

\noindent This includes an R package for the various functions used for testing and plotting the figures.
All plots are done using ggplot2 from \cite{ggplot2}.
The phase-type distributions are computed using the \textit{actuaR} package from \cite{actuar2008}.